%%%% IACR Transactions TEMPLATE %%%%
% This file shows how to use the iacrtrans class to write a paper.
% Written by Gaetan Leurent gaetan.leurent@inria.fr (2020)
% Public Domain (CC0)


%%%% 1. DOCUMENTCLASS %%%%
\documentclass{iacrtrans}
%%%% NOTES:
% - Change "journal=tosc" to "journal=tches" if needed
% - Change "submission" to "final" for final version
% - Add "spthm" for LNCS-like theorems


%%%% 2. PACKAGES %%%%
\usepackage[utf8]{inputenc} % UTF-8 encoding
\usepackage[T1]{fontenc}  % Font encoding
\usepackage{lmodern}
\usepackage{graphicx}     % For including images
\usepackage{geometry}     % For page layout
\usepackage{xcolor}       % For custom colors
\usepackage{titlesec}     % For title formatting
\usepackage{fancyhdr}     % For headers and footers
\usepackage{tcolorbox}    % For colored boxes
\usepackage{hyperref}
\usepackage{xspace}    	 % For new commands
\usepackage{tocloft}
\usepackage[longend]{algorithm2e}
\usepackage{textgreek}
\usepackage{mathtools}
\usepackage{booktabs}
\usepackage{multirow}
% For {x/c}mark
\usepackage{amssymb}
\usepackage{pifont}
% For bar plots
\usepackage{pgfplots}
\pgfplotsset{width=7cm,compat=1.8}

\newcommand{\cmark}{\ding{51}}%
\newcommand{\xmark}{\ding{55}}%

% Macros
\newcommand{\BLS}{\textsf{BLS}\xspace}
\newcommand{\XMSS}{\textsf{XMSS}\xspace}
\newcommand{\WOTS}{\textsf{WOTS}\xspace}
\newcommand{\PRF}{\textsf{PRF}\xspace}
\newcommand{\KECCAK}{\textsf{SHA3}\xspace}
\newcommand{\Poseidon}{\textsf{Poseidon}\xspace}
\newcommand{\PoseidonTwo}{\textsf{Poseidon2}\xspace}
\newcommand{\Hydra}{\textsf{Hydra}\xspace}
\newcommand{\PoseidonPi}{$\mathsf{Poseidon}^{\pi}$\xspace}
\newcommand{\PoseidonTwoPi}{$\mathsf{Poseidon2}^{\pi}$\xspace}


\newtcolorbox{algostyle}[1][]{%
    colback=gray!10, % Light gray background
    colframe=black, % Black border
    boxrule=1pt, % Thin border
    rounded corners,
    #1
}


%%%% 3. AUTHOR, INSTITUTE %%%%
\author{Alexandre Adomnic\u{a}i}%\inst{1}}
\institute{
  DV Labs, \email{alexandre@dvlabs.tech}
  %\and
  %Institute B, City, Country, \email{john@institute}
}
%%%% NOTES:
% - We need a city name for indexation purpose, even if it is redundant
%   (eg: University of Atlantis, Atlantis, Atlantis)
% - \inst{} can be omitted if there is a single institute,
%   or exactly one institute per author


%%%% 4. TITLE %%%%
\title{Towards Threshold Hash-Based Signatures for Post-Quantum Distributed Validators}
%%%% NOTES:
% - If the title is too long, or includes special macro, please
%   provide a "running title" as optional argument: \title[Short]{Long}
% - You can provide an optional subtitle with \subtitle.

\begin{document}

\maketitle


%%%% 5. KEYWORDS %%%%
\keywords{MPC \and Hash-based signatures}


%%%% 6. ABSTRACT %%%%
\begin{abstract}
With recent advances in quantum computing, post-quantum cryptographic algorithms are being actively deployed in real-world applications.
For Ethereum, a transition to post-quantum cryptography would require replacing many primitives, including the \textsf{BLS12-381} signature schemes used by validators on the beacon chain.
Because \BLS leverages its bilinear pairing property to aggregate multiple validator signatures, enabling both performance improvements and space savings, its replacement presents a particular challenge.
Furthermore, the bilinearity of the pairing function also enables straightforward threshold signatures, which are fundamental to distributed validator solutions.
The Ethereum foundation recently introduced hash-based signature schemes as post-quantum alternatives to \BLS. 
In this research, we study the practical challenges of deploying such schemes in a distributed manner.
\end{abstract}


% Main content
\section{Introduction}
If a cryptographically relevant quantum computer is built, Shor’s algorithm will pose serious threats to traditional public key cryptosystems based on large number factorization and (elliptic curve) discrete logarithm problems, such as \textsf{RSA} and \textsf{ECDSA}.
As a response, cryptographers are developing new algorithms that offer security even against an attacker equipped with quantum computers, denoted as post-quantum cryptography (PQC).
There are several standardization processes ongoing, notably by the one NIST which has already published a set of standards: ML-KEM~\cite{fips203} as key encapsulation mechanism along with \textsf{ML-DSA}~\cite{fips204}, \textsf{SLH-DSA}~\cite{fips205} and \textsf{FN-DSA}, to be published soon, as signature schemes.
Because \textsf{ML-DSA} and \textsf{FN-DSA} are both lattice-based, NIST has sollicited the submission of additional signatures schemes to expand its PQC signature portfolio\footnote{\url{https://csrc.nist.gov/projects/pqc-dig-sig}}.
Unfortunately, the current post-quantum signature schemes selected by NIST for standardization do not inherently support advanced functionalities such as signature aggregation and/or threshold signing.
Signature aggregation is commonly used in blockchain systems as this powerful feature allows to compress many signatures into a short aggregate, shrinking the storage space and speeding-up the verification time.
Ethereum leverages  aggregate signatures in its consensus layer thanks to the \BLS signature scheme~\cite{bls2001}.
On top of intrinsically supporting signature/public key aggregation, \BLS is straightforward to be turned into a threshold signature scheme when combined with Shamir secret sharing which lends itself to Distributed Validator Technology (DVT)~\cite{DVT}.
%\begin{itemize}
%	\item A shared secret $s$ is shared (or distributively generated) in $n$ parts $s_0, \cdots, s_i$ using a $t$-out-of-$n$ setting (i.e. at least $t$ must collude to produce a valid signature under $s$).
%	\item To sign a message $M$, each party $\mathcal{P}_i$ \textit{partially} signs $M$ using its own private share $s_i$ as in non-threshold BLS and sends its partial signature along with its partial public key to the aggregator.
%	\item Once (at least) $t$ partial signatures have been collected and verified, the aggregator combines the partial signatures all together using Lagrange interpollation to get a \textit{full} signature of $M$ under $s$.
%\end{itemize}
An important observation in the case of \BLS is that aggregated and/or threshold \BLS signatures are indistinguishable from raw ones, all being points on the same elliptic curve.
This allows to build efficient DVT middleware solutions, such as the \texttt{charon}\footnote{\url{https://github.com/ObolNetwork/charon}}, which operates in a totally transparent manner from a consensus client point of view.
%This allows to build distributed validator middleware solutions such as \href{https://github.com/ObolNetwork/charon}{\texttt{charon}}, which sits in between a consensus client %(also called beacon node) and a validator client.
%In this configuration, the validator client is in charge of signing duties while the middleware  
%Interestingly in that case threshold signatures operate in the shades, without the knowledge of the upstream and downstream clients.
However, because \BLS is based on elliptic curve pairing, it would not provide enough security against quantum adversaries.
To address this concern, the Ethereum foundation recently introduced a family of hash-based signature schemes as post-quantum atlernatives to \BLS~\cite{cryptoeprint:2025/055}.
The main idea behind their design is to aggregate hash-based signatures using post-quantum succinct non-interactively argument of knowledge (pqSNARK) systems. %, an approach which has already been explored in a previous work~\cite{agg-hash-based-starks}.
While this seems to be a promising alternative, it would have considerable impacts on distributed validators solutions which currently rely on the homomorphic properties of \BLS to leverage threshold signatures.
The goal of this document is to identify the challenges that could arise from such a transition and discuss the potential solutions to address them.



\section{Aggregate hash-based signatures using SNARKs}
\subsection{Hash-based signatures}
As their name suggests, hash-based signature schemes rely on hash functions as their core primitive.
In contrast to public key cryptosystems, there is no strong evidence that symmetric cryptography, including hash functions, would be significantly impacted by quantum computers.
Although recommendations on symmetric cryptography may vary between cybersecurity agencies\footnote{ANSSI recommends at least the same security as \textsf{AES-256} and \textsf{SHA2-384} for block ciphers and hash functions, respectively, whereas NIST, NCSC and BSI recommend \textsf{AES-128} and \textsf{SHA-256}.}, hash-based signatures are seen as a conservative choice for post-quantum security given their well-understood security.
%Although quantum computers theoretically improve the birthday attack complexity from $\mathcal{O}(\sqrt{n})$ to $\mathcal{O}(n^{1/3})$, Bernstein highlighted that it would not be the case in practice as it does not take into account the practical costs of quantum collision search~\cite{quant_hash_collision}.
%Because hash functions are well understood now being are part of the cryptographic landscape for half a century
The classical approach to build hash-based signatures is to combine many one-time signature (OTS) key pairs into a Merkle tree~\cite{10.5555/909000} whose root serves as the many-time public key.
To provide a concrete example, we hereafter introduce the Winternitz OTS (\WOTS) scheme. 
%To provide a concrete instantiation, we hereafter introduce the \XMSS scheme which is based on Winternitz OTS (\WOTS).

\paragraph{Winternitz OTS.}
\WOTS is parameterized by two values:
\begin{itemize}
	\item the Winternitz parameter $w$, being a power of 2.% (usually selected from $\{2,4,8\}$).
	\item a $n$-bit hash function $H$ such that $n=vw$
\end{itemize}
To generate an OTS key pair, one randomly generates $v$ $n$-bit secret keys  $sk_0,\cdots,sk_{v-1}$ and derives the corresponding public keys using hash chains of length $2^w-1$ (\textit{i.e.}, $pk_i = H^{2^w-1}(sk_i)$).
To sign a message $m$, a checksum over $m$ is appended to it before hashing.
The $n$-bit output is then divided into $v$ $w$-bit chunks $c_0,\cdots,c_{v-1}$ and the signature consists of $\sigma = \sigma_0, \cdots, \sigma_{v-1}$ where $\sigma_i = H^{c_i}(sk_i)$. %for $i \in \{0,\cdots,v-1\}$.
To verify a signature, one checks that $H^{2^w-1-c_i}(\sigma_i) = pk_i$ for $i \in \{0,\cdots,v-1\}$.
%\footnote{Practical \WOTS deployments actually do not generate the private keys randomly but rely on a pseudorandom function (PRF) with a secret seed instead to reduce memory storage requirements.}

%To sign the $i$th message, the signer uses the $i$th OTS secret key and includes the corresponding public key along with its Merkle path in the signature.
%This introduces the concept statefulness: because security depends on the unique usage of each OTS key pair, it is crucial to keep track of which keys have already been used.
%As a corollary, this state management also introduces the concept of lifetime since once all the OTS keys have been used, it is not possible to sign anymore.

\paragraph{Merkle tree.}
To build a many-time signature scheme from \WOTS, one can combine multiple key pairs with a binary tree where each node is the hash of its children, commonly referred to as Merkle tree.
For a height parameter $h$, such a tree is built from $2^h$ leaves $l_0,\cdots,l_{2^h-1}$,  each being the hash of a \WOTS public key (\textit{i.e.}, $l_i = H(pk_{i_0}, \cdots, pk_{i_{v-1}})$).
The root constitutes the many-time public key and commits to all OTS public keys.
Note that to reduce memory requirements in practice, it is recommended to generate the \WOTS secret keys using a pseudorandom function (PRF) rather than using a random number generator~\cite{RFC8391}.
To sign the $i$th message, the signer uses the $i$th OTS secret key and includes the Merkle path of the corresponding public key in the signature.
To verify the signature, the verifier computes the public key from \WOTS signature and then, thanks to the Merkle path, verifies that its digest is indeed the leaf at position $i$.
This introduces the concept statefulness: because security depends on the unique usage of each OTS key pair, it is crucial to keep track of which keys have already been used.

\subsection{SNARK-based aggregation}
The idea behind SNARK-based aggregation is for an aggregator to turn individual signatures, possibly over different messages, into a SNARK proof attesting their validity.
Note that this principle can be used to thresholdize a signature scheme: given a $k$-of-$n$ setting, the aggregator can generate a proof attesting that it verified $k$ distinct signatures over the same message and that signers are part of the quorum.  
A valuable feature of this approach is its non-interactiveness: the aggregator only needs to collect individual signatures in order to compute the proof, without any additional communication.
Combining such a construction with hash-based signatures has been first explored by Khaburzaniya \textit{et al.}~\cite{agg-hash-based-starks}, using \WOTS with 1-bit chunks (instantiated with the \textsf{Rescue-Prime} hash function) along with STARKs.
To complement this research, the work from Drake \textit{et al.}~\cite{cryptoeprint:2025/055} does not focus on a specific hash-based signature scheme but explores a variety of tradeoffs by introducing a generalized variant of \XMSS~\cite{10.1007/978-3-642-25405-5_8} and providing security proofs that hold for all its instantiations.
Notably, their security proofs do not model hash functions as random oracles and rely on standard model properties instead, such as preimage/collision resistance, providing concrete security targets. 


\section{Towards threshold XMSS}
The downside of building a threshold hash-based signature by leveraging a SNARK system as mentioned above is that the aggregation of threshold signatures would not be straightforward (since threshold signatures are proofs instead of raw hash-based signatures).
In the case of the Beacon chain where threshold signatures occur \textit{before} aggregation duties, it is imperative for distributed validator middlewares to output signatures that can be aggregated according to the protocol.
Therefore, this section focuses on constructions which lead to threshold Winternitz signatures that are indistinguishable from non-threshold ones, as in \BLS.
\subsection{Distributed hash-based signatures with Boolean shares.}
Distributed variants of hash-based signatures, including \XMSS, have been explored by Kelsey, Lang and Lucks in~\cite{cryptoeprint:2022/241} where they introduce $n$-of-$n$ and $k$-of-$n$ threshold signature schemes which rely on Boolean shares.
%\paragraph{$n$-of-$n$ setting.}
For the $n$-of-$n$ setting, a trusted dealer starts from an existing Merkle tree and splits each \WOTS secret key $\mathsf{sk_i}$ by generating $n$ random values $\mathsf{r}^0_i,\cdots,\mathsf{r}^{n-1}_i$ to compute $\mathsf{r}^h_i = \mathsf{r}^0_i \oplus \mathsf{r}^1_i \oplus \cdots \oplus \mathsf{r}^{n-1}_i \oplus \mathsf{sk}_i$.
This introduces an additional party called the \textit{helper} whose role is to store and provide the relevant helper shares whenever required.
That way, to produce a \WOTS using for $\mathsf{sk}_i$, each party can sign independently using its Boolean key share assuming the aggregator has access to the helper share $r^h_i$.
Note that it has to be done for each component of the secret key: assuming a \WOTS scheme to sign $n=vw$-bit messages, it means that each \WOTS key requires $v2^w-1$ helper shares.
Furthermore, the trusted dealer also needs to provide the helper with  shares for each Merkle paths, leading to high memory requirement for the helper overall.
%\mto generate random values $n$ random values and to exclusive-OR them
To minimize memory usage for the parties, the key shares are actually generated pseudorandomly using a \PRF as detailed in Algorithm~\ref{alg:bool_split_n}.
To turn their $n$-of-$n$ scheme into a $k$-of-$n$ threshold scheme, they propose to instantiate a Merkle tree that contains keys for all possible $\binom{n}{k}$ quorums.
Beyond complexity, this increases the height of the Merkle tree and hence the signature size as well as the memory requirements for the helper.
Because their designs are incompatible with distributed key generation (DKG) methods, we investigate other solutions instead. 

%\begin{myalgorithm}{Split a Merkle tree of WOTS keys into distributed key shares for $n$-of-$n$ signatures, according to~\cite{cryptoeprint:2022/241}.}

\begin{algorithm}[htbp]
    \caption{Split a Merkle tree of \WOTS keys into distributed key shares for $n$-of-$n$ signatures, according to~\cite{cryptoeprint:2022/241}.}
    \begin{algostyle} % Apply styling box inside
Input parameters:
\vspace{-.75em}
\begin{itemize}
\itemsep-.5em
\item Merkle tree built out of a $n$-bit hash function $H$ and $2^h$ \WOTS secret keys $\mathsf{sk}_0,\cdots,\mathsf{sk}_{2^h-1}$ to sign $n=vw$-bit messages (\textit{i.e.}, $\mathsf{sk}_i = (\mathsf{sk}_{i,0}, \cdots, \mathsf{sk}_{i,v-1})$).
\item A pseudorandom function $\mathsf{PRF}_K(x,l)$ parametrized by a $k$-bit key $K$ which takes as input a seed $x$ along with the output bit length $l$.
\item A set of distributed parties $\mathcal{P}$.
\end{itemize}

Output parameters:
\vspace{-.75em}
\begin{itemize}
\itemsep-.5em
\item Secret keys $\mathsf{key}_p$ for each party $p \in \mathcal{P}$.
\item Helper shares $\mathsf{sk}^h_{i,j}$ and $\mathsf{path}^h_{i}$ for $i \in \{0,\cdots,2^h-1\}$ and $j \in \{0, \cdots, v-1\}$.
\end{itemize}

\tcp{picks secrets at random for each party}
\ForEach(){$p \in \mathcal{P}$} {
	$\mathsf{key}_p \xleftarrow[]{\$} \{0,1\}^k $ 
}\vspace{1em}

%\For(\tcp*[f]{for each leaf}){$j=0$ \KwTo $2^h$} {
\tcp{builds Merkle path helper shares}
\For(){$i=0$ \KwTo $2^h-1$} {
	$\mathsf{path}^h_i \leftarrow \mathsf{path}_i$\\
	\ForEach(){$p \in \mathcal{P}$} {
		$\mathsf{path}^h_i \leftarrow \mathsf{path}^h_i \oplus \mathsf{PRF}_{\mathsf{key}_p}\big((\mathsf{domain_{path}}, i), nh\big)$
	}
}\vspace{1em}
	
\tcp{builds WOTS key helper shares}
\For(\tcp*[f]{for each WOTS secret key}){$i=0$ \KwTo $2^h-1$} {
	\For(\tcp*[f]{for each key component}){$j=0$ \KwTo $v-1$} {
		\For(\tcp*[f]{for each w-bit chunk}){$c=0$ \KwTo $2^w-1$} {
		$\mathsf{sk}^h_{i,j}[c] \leftarrow H^c(\mathsf{sk}_{i,j})$ \\
		\ForEach(){$p \in \mathcal{P}$} {
			$\mathsf{sk}^h_{i,j}[c] \leftarrow \mathsf{sk}^h_{i,j}[c] \oplus \mathsf{PRF}_{\mathsf{key}_p}\big((\mathsf{domain_{key}}, i, j, c), n\big)$
			}
		}
	}

}
\label{alg:bool_split_n}
\end{algostyle}
\end{algorithm}

\subsection{Leveraging secret-sharing}
Another approach is to leverage a secret-sharing scheme to split \WOTS secret keys into shares distributed among participants.
However, it requires to jointly compute all hash function calls in a multi-party computation (MPC) setting.
%However, the difficulty arises when Merkle trees come into play: computing Merkle paths from secret key shares requires to process all hash function calls in a multi-party computation (MPC) fashion.
This can be very challenging in practice, especially in low latency scenarios such as performing validator duties on Ethereum, as the time to produce a threshold signature may largely exceed the requirements (see \textit{e.g.} the work from Cozzo and Smart~\cite{sharing_luov19} which estimates around 85 minutes to compute a threshold \textsf{SPHINCS+} signature with \textsf{SHA-3} as the underlying hash function).
A possible workaround could be to evaluate the hash function calls in a distributed manner during key generation, so that each party can store a share for each component of each \WOTS secret key.
Note that this would require to be the done for all Merkle paths as well.
Assuming a Merkle tree instantiated with a 256-bit hash function and $2^{18}$ \WOTS keys with a parameter $w = 4$, that would lead to a memory requirement of $\approx 4.5$ gigabytes.
A comprehensive performance analysis of MPC hash functions is crucial to assess what time-memory tradeoffs (\textit{i.e.}, hashing \WOTS secret key components on-the-fly instead of precomputing the value for each chunk) would be viable in practice.

%



%Bear in mind that avoiding any MPC hash invocation implies that \WOTS secret keys cannot be generated pseudo-randomly from a secret seed shared amongst participants, further increasing memory consumption.
 
%Still, even if all parties store their own \WOTS secret key shares and corresponding Merkle paths, they cannot sign non-interactively.
%Indeed, \XMSS not being a deterministic signature scheme, the signers have to agree on a random number to be included in the randomized message digest calculation.


\section{Hash functions over MPC}
%\subsection{MPC-friendly hash functions}
Traditional hash functions such as \KECCAK operate over binary fields to enable efficient implementations in both hardware and software on a wide range of platforms.
However, they lead to poor performance when employed within advanced cryptographic protocols such as MPC.
This is mainly due to the fact that traditional schemes are designed to minimize their overall gate count without minimizing specifically nonlinear gates\footnote{They are actually symmetric designs that aim at minimizing the number of nonlinear gates for efficient software masked implementations against side-channel attacks, see for instance~\cite{fse-2014-27573}.} which require communication between parties in an MPC setting, unlike linear gates that can be computed locally.
The overload induced by these communications is such that it can constitute the bottleneck in MPC protocols, as highlighted by an attempt to thresholdize PQC signatures schemes~\cite{sharing_luov19}.
In response, new primitives with design constraints finely tuned for advanced cryptographic protocols have emerged, known as \textit{arithmetization-oriented} primitives.
They usually operate over $\mathbb{F}_p$ with $p$ prime, making them natively compatible with linear secret sharing schemes, and rely on multiplications for nonlinear operations.
Among them, \Poseidon~\cite{poseidon} has found its place into many Ethereum applications thanks to its efficiency in verifiable computing and its successor \PoseidonTwo~\cite{poseidon2} is currently being considered for Ethereum protocols that rely on zero-knowledge proofs\footnote{\url{https://www.poseidon-initiative.info/}}.
\subsection{The Poseidon2 family of hash functions}
%Because the generalized \XMSS scheme introduced in~\cite{} has been instantiated with \PoseidonTwo for benchmarking purposes, we will use this hash function for 
%\paragraph{Poseidon2}
\paragraph{Overview.}
\PoseidonTwo is built upon the \PoseidonTwoPi permutation operating over $\mathbb{F}_p^t$ with $p > 2^{30}$ prime and $t \in \{2,3,4,8,12,16,20,24\}$. %such that $t = c + r$ where $c$ and $r$ refer to the capacity and rate of the sponge construction, respectively. %being Substitution-Permutation Networks (SPN) based on the HADES design strategy.
The permutation is meant to be combined with either a compression function or a sponge construction to build a hash function.
\PoseidonTwoPi is based on the HADES design strategy which makes a distinction between external and internal rounds.
Internal rounds (also called partial rounds) apply the nonlinear layer to only a part of the state, usually a single element, whereas external rounds (also called full rounds) process all elements in the same way.
More precisely, \PoseidonTwoPi processes an internal state $x = (x_0,\cdots,x_{t-1}) \in \mathbb{F}_p^t$ as follows:
\begin{align*}
 \mathsf{Poseidon2}^{\pi}(x) = \mathcal{E}_{R_F-1} \circ \cdots \circ \mathcal{E}_{R_F/2} \circ \mathcal{I}_{R_P-1} \circ \cdots \circ \mathcal{I}_{0}\circ \mathcal{E}_{R_F/2-1} \circ \cdots \circ \mathcal{E}_{0}(M_{\mathcal{E}} \cdot x) \,
\end{align*}
where $\mathcal{E}$ and $\mathcal{I}$ refer to external and internal round functions iterated for $R_F$ and $R_P$ rounds, respectively.
Note that a linear layer is applied before running the first external round, which differs from the original \PoseidonPi design.
The external/full round function is defined by:
\begin{align*}
 \mathcal{E}(x) = M_{\mathcal{E}} \cdot \Big(\big(x_0+c_0^{(i)}\big)^d,\cdots,\big(x_{t-1}+c_{t-1}^{(i)}\big)^d\Big) \,
\end{align*}
where $d \geq 3$ is the smallest integer such that gcd$(d,p-1) = 1$, $M_{\mathcal{E}}$ is a $t \times t$ maximum distance separable (MDS) matrix and $c_j^{(i)}$ is the $j$-th round constant for the $i$-th external round.
The internal/partial round function is defined by:
\begin{align*}
 \mathcal{I}(x) = M_{\mathcal{I}} \cdot \Big(\big(x_0+\hat{c}_0^{(i)}\big)^d,x_1,\cdots,x_{t-1}\Big) \,
\end{align*}
where $d \geq 3$ as before, $M_{\mathcal{I}}$ is a $t \times t$ MDS matrix and $\hat{c}_0^{(i)}$ is the round constant for the $i$-th internal round.


\paragraph{Efficient instantiations for hash-based signatures over MPC.}
Since \PoseidonTwo is a generic construction, all instantiations will most likely not provide the same level of MPC-friendliness.
Because all operations but exponentiations can be computed locally in an MPC setting, one should aim at minimizing the $d$ parameter as it would result in fewer multiplications.
From a permutation-only perspective, one would also be tempted to minimize $t$ and $R = R_F + R_P$ parameters  as the amount of exponentiations is directly derived from them.
However, at the hash function level, the optimal parameter selection depends on the input size to be processed. Indeed for large inputs that require a sponge mode as the underlying construction, having a large rate would allow to absorb more data per permutation, and eventually leading to fewer calls and fewer exponentiations in the end.
In the case of hash-based signatures, most hash calls process small inputs to compute either hash chains from secret keys or nodes in the Merkle tree, with the exception of leafs which are obtained by hashing multiple public keys.
This is why the generalized \XMSS scheme from~\cite{cryptoeprint:2025/055} instantiates \PoseidonTwo with the compression mode for chain and tree hashing, whereas it uses the sponge mode for leaf hashing.
For hash-based signatures over MPC however, one can disregard the specific case of leaf hashing: since all inputs are public it is possible to recombine the shared values together in order to calculate the hash in a non-distributed manner.
Therefore, we focus solely on instantiations based on the compression mode.
More specifically, instantiations from~\cite{cryptoeprint:2025/055} all consider a 31-bit prime field for efficient SNARK-based aggregation with $t = 16$ and $t = 24$ for chain and tree hashing, respectively.
Table~\ref{tab:poseidon2_inst31} lists the relevant parameters for two such primes, namely Mersenne31 and BabyBear, which enable highly efficient implementation techniques.
%The remaining parameters of interest, namely $d$ and $R = R_F + R_P$, depend on the value of $p$ as detailed in 




\begin{table}[htbp]
	\centering
	\begin{tabular}{ccccc}
		\toprule
    		{$p$} & {$t$} & {$d$} & {$R_F$} & {$R_P$} \\
    		\midrule
    		& 16 & 5 & 8 & 14 \\
    		\multirow{-2}{*}{$2^{31}-1$} & 24 & 5 & 8 & 22 \\
    		\midrule
    		& 16 & 7 & 8 & 13 \\
    		\multirow{-2}{*}{$2^{31}-2^{27} + 1$}  & 24 & 7 & 8 & 21 \\
    		\bottomrule
	\end{tabular}
	\caption{\PoseidonTwo parameters for 31-bit prime fields.\label{tab:poseidon2_inst31}}
\end{table}

\subsection{Secure multiplication}

\section{MPC protocols for distributed validators}
MPC protocols greatly differ based on their security, adversarial and network assumptions. This section aims at identifying the relevant properties in the case of distributed validators in order to shortlist the most meaningful protocolsS.
\subsection{Security model}
\paragraph{Adversarial structure.}
Let $n$ denote the number of participating parties and let $t$ denote a bound on the number of parties that may be corrupted.
MPC protocols are usually designed to provide security in either the dishonest (i.e., $t < n$), honest (i.e., $t < n/2$) or two-thirds honest (i.e., $t < n/3$) majority settings.
While dishonest-majority protocols offer the strongest security guarantees, they come at a significant cost.
Restricting the model to an honest majority, however, enables substantial performance improvements.
Thus, it is crucial to determine whether a dishonest-majority setting is truly necessary to achieve optimal efficiency.
Regarding distributed validators, it is important to note that they rely not only on a threshold signature scheme (TSS) but also on a consensus algorithm, which ensures that all parties agree on the same message to be  signed.
%Therefore, it is important to clearly state the
 
\paragraph{Adversarial behaviour.}

\paragraph{Security assumption.}

\section{TODO}

\begin{enumerate}
	\item Investigate on the generation of Beaver triples, both offline \& online for the MASCOT protocol
	\item Investigate on the memory usage of Beaver triples in the context of MASCOT
	\item Benchmark distributed hash in a "real" setting (i.e. not in a LAN network) in both serial and parallel ways
	\item Investigate whether we should rely on malicious security for both DKG and signature generation or if we could consider a more relaxed security model
	\item Add details on Sbox optimization to save communication rounds (i.e. cubes)
	%\item Specify somewhere that although all hash chains are precomputed, interaction is still needed to agree on random
\end{enumerate}

%%%% 8. BILBIOGRAPHY %%%%
\bibliographystyle{alpha}
\bibliography{bib}
%%%% NOTES
% - Download abbrev3.bib and crypto.bib from https://cryptobib.di.ens.fr/
% - Use biblio.bib for additional references not in the cryptobib database.
%   If possible, take them from DBLP.

\end{document}
