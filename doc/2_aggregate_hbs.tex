\section{Aggregate hash-based signatures using SNARKs}
\subsection{Hash-based signatures\label{sec:hbs}}
As their name suggests, hash-based signature schemes rely on hash functions as their core primitive.
In contrast to public key cryptosystems, there is no strong evidence that symmetric cryptography, including hash functions, would be significantly impacted by quantum computers.
Although recommendations on symmetric cryptography may vary between cybersecurity agencies\footnote{ANSSI recommends at least the same security as \textsf{AES-256} and \textsf{SHA2-384} for block ciphers and hash functions, respectively, whereas NIST, NCSC and BSI recommend \textsf{AES-128} and \textsf{SHA-256}.}, hash-based signatures are seen as a conservative choice for post-quantum security given their well-understood security.
%Although quantum computers theoretically improve the birthday attack complexity from $\mathcal{O}(\sqrt{n})$ to $\mathcal{O}(n^{1/3})$, Bernstein highlighted that it would not be the case in practice as it does not take into account the practical costs of quantum collision search~\cite{quant_hash_collision}.
%Because hash functions are well understood now being are part of the cryptographic landscape for half a century
The classical approach to build hash-based signatures is to combine many one-time signature (OTS) key pairs into a Merkle tree~\cite{10.5555/909000} whose root serves as the many-time public key.
To provide a concrete example, we hereafter introduce the Winternitz OTS (\WOTS) scheme. 
%To provide a concrete instantiation, we hereafter introduce the \XMSS scheme which is based on Winternitz OTS (\WOTS).

\paragraph{Winternitz OTS.}
\WOTS is parameterized by two values:
\begin{itemize}
	\item the Winternitz parameter $w$, being a power of 2.% (usually selected from $\{2,4,8\}$).
	\item a $n$-bit hash function $H$ such that $n=vw$
\end{itemize}
To generate an OTS key pair, one randomly generates $v$ $n$-bit secret keys  $sk_0,\cdots,sk_{v-1}$ and derives the corresponding public keys using hash chains of length $2^w-1$ (\textit{i.e.}, $pk_i = H^{2^w-1}(sk_i)$).
To sign a message $m$, a checksum over $m$ is appended to it before hashing.
The $n$-bit output is then divided into $v$ $w$-bit chunks $c_0,\cdots,c_{v-1}$ and the signature consists of $\sigma = \sigma_0, \cdots, \sigma_{v-1}$ where $\sigma_i = H^{c_i}(sk_i)$. %for $i \in \{0,\cdots,v-1\}$.
To verify a signature, one checks that $H^{2^w-1-c_i}(\sigma_i) = pk_i$ for $i \in \{0,\cdots,v-1\}$.
%\footnote{Practical \WOTS deployments actually do not generate the private keys randomly but rely on a pseudorandom function (PRF) with a secret seed instead to reduce memory storage requirements.}

%To sign the $i$th message, the signer uses the $i$th OTS secret key and includes the corresponding public key along with its Merkle path in the signature.
%This introduces the concept statefulness: because security depends on the unique usage of each OTS key pair, it is crucial to keep track of which keys have already been used.
%As a corollary, this state management also introduces the concept of lifetime since once all the OTS keys have been used, it is not possible to sign anymore.

\paragraph{Merkle tree.}
To build a many-time signature scheme from \WOTS, one can combine multiple key pairs with a binary tree where each node is the hash of its children, commonly referred to as Merkle tree.
For a height parameter $h$, such a tree is built from $2^h$ leaves $l_0,\cdots,l_{2^h-1}$,  each being the hash of a \WOTS public key (\textit{i.e.}, $l_i = H(pk_{i_0}, \cdots, pk_{i_{v-1}})$).
The root constitutes the many-time public key and commits to all OTS public keys.
Note that to reduce memory requirements in practice, it is recommended to generate the \WOTS secret keys using a pseudorandom function (PRF) rather than using a random number generator~\cite{RFC8391}.
To sign the $i$th message, the signer uses the $i$th OTS secret key and includes the Merkle path of the corresponding public key in the signature.
To verify the signature, the verifier computes the public key from \WOTS signature and then, thanks to the Merkle path, verifies that its digest is indeed the leaf at position $i$.
This introduces the concept statefulness: because security depends on the unique usage of each OTS key pair, it is crucial to keep track of which keys have already been used.

\subsection{SNARK-based aggregation}
The idea behind SNARK-based aggregation is for an aggregator to turn individual signatures, possibly over different messages, into a SNARK proof attesting their validity.
Note that this principle can be used to thresholdize a signature scheme: given a $k$-of-$n$ setting, the aggregator can generate a proof attesting that it verified $k$ distinct signatures over the same message and that signers are part of the quorum.  
A valuable feature of this approach is its non-interactiveness: the aggregator only needs to collect individual signatures in order to compute the proof, without any additional communication.
Combining such a construction with hash-based signatures has been first explored by Khaburzaniya \textit{et al.}~\cite{agg-hash-based-starks}, using \WOTS with 1-bit chunks (instantiated with the \textsf{Rescue-Prime} hash function) along with STARKs.
To complement this research, the work from Drake \textit{et al.}~\cite{cryptoeprint:2025/055} does not focus on a specific hash-based signature scheme but explores a variety of tradeoffs by introducing a generalized variant of \XMSS~\cite{10.1007/978-3-642-25405-5_8} and providing security proofs that hold for all its instantiations.
Notably, their security proofs do not model hash functions as random oracles and rely on standard model properties instead, such as preimage/collision resistance, providing concrete security targets. 
