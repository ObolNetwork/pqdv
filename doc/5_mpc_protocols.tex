\section{MPC protocols for distributed validators}
MPC protocols greatly differ based on their security, adversarial and network assumptions. This section aims at identifying the relevant properties in the case of distributed validators in order to shortlist the most meaningful protocolsS.
\subsection{Security model}
\paragraph{Adversarial structure.}
Let $n$ denote the number of participating parties and let $t$ denote a bound on the number of parties that may be corrupted.
MPC protocols are usually designed to provide security in either the dishonest (i.e., $t < n$), honest (i.e., $t < n/2$) or two-thirds honest (i.e., $t < n/3$) majority settings.
While dishonest-majority protocols offer the strongest security guarantees, they come at a significant cost.
Restricting the model to an honest majority, however, enables substantial performance improvements.
Thus, it is crucial to determine whether a dishonest-majority setting is truly necessary to achieve optimal efficiency.
Regarding distributed validators, it is important to note that they rely not only on a threshold signature scheme (TSS) but also on a consensus algorithm\footnote{\url{https://github.com/ethereum/distributed-validator-specs}}, which ensures that all parties agree on the same message to be  signed.
It is well known that, in an asynchronous network, Byzantine fault tolerance (BFT) can only be achieved if $t < n/3$~\cite{10.1145/322186.322188}.
Therefore, considering a two-thirds honest majority for the MPC computation of the signature aligns well with BFT assumptions.
%However, if crash fault tolerance (CFT) is sufficient instead of BFT, the fault tolerance threshold may increase and considering an honest majority would allow the system to tolerate more faulty parties.
 
\paragraph{Adversarial behaviour.}
In the MPC litterature, corrupted parties are either considered malicious/active if they behave arbitrarily (\textit{i.e.}, they can deviate from the protocol) or semi-honest/passive if they follow the protocol but combine their respective information to learn more than they should be allowed to.
To ensure general compatibility with BFT consensus protocols, we aim for malicious security.
Still, protocols in the semi-honest model can be of interest for their efficiency if crash fault tolerance (CFT) suffices.

\paragraph{Security assumption.}
The security of an MPC protocol can rely on different assumptions. It can be \textit{computational} secure, meaning that it depends on the hardness of specific mathematical problems (\textit{e.g.}, factoring large numbers), or it can be \textit{information-theoretic} secure, meaning that it achieves security based on principles of information theory without relying on computational hardness.
Since we aim at computing a post-quantum signature scheme in an MPC fashion, we aim for an information-theoretic secure MPC protocol as it is not vulnerable against quantum computers.
%While the information-theoretic setting provides the highest security, it is not always possible to achieve and requires an honest majority.

\subsection{Protocol candidates}
Below we list MPC protocols that achieve information-theoretic malicious security in the (two-thirds) honest majority setting.

\paragraph{ATLAS.}

\paragraph{TurboPack.}