\section{Hash functions over MPC}
Traditional hash functions such as \KECCAK operate over binary fields to enable efficient implementations in both hardware and software on a wide range of platforms.
However, they lead to poor performance when employed within advanced cryptographic protocols such as MPC.
This is mainly due to the fact that traditional schemes are designed to minimize their overall gate count without minimizing specifically nonlinear gates\footnote{They are actually symmetric designs that aim at minimizing the number of nonlinear gates for efficient software masked implementations against side-channel attacks, see for instance~\cite{fse-2014-27573}.} which require communication between parties in an MPC setting, unlike linear gates that can be computed locally.
The overload induced by these communications is such that it can constitute the bottleneck in MPC protocols, as highlighted by an attempt to thresholdize PQC signatures schemes~\cite{sharing_luov19}.
In response, new primitives with design constraints finely tuned for advanced cryptographic protocols have emerged, known as \textit{arithmetization-oriented} primitives.
They usually operate over $\mathbb{F}_p$ with $p$ prime, making them natively compatible with linear secret sharing schemes, and rely on multiplications for nonlinear operations.
Among them, \Poseidon~\cite{poseidon} has found its place into many Ethereum applications thanks to its efficiency in verifiable computing and its successor \PoseidonTwo~\cite{poseidon2} is currently being considered for Ethereum protocols that rely on zero-knowledge proofs\footnote{\url{https://www.poseidon-initiative.info/}}.
\subsection{The Poseidon2 family of hash functions}
\paragraph{Overview.}
\PoseidonTwo is built upon the \PoseidonTwoPi permutation operating over $\mathbb{F}_p^t$ with $p > 2^{30}$ prime and $t \in \{2,3,4,8,12,16,20,24\}$.
The permutation is meant to be combined with either a compression function or a sponge construction to build a hash function.
\PoseidonTwoPi is based on the HADES design strategy which makes a distinction between external and internal rounds.
Internal rounds (also called partial rounds) apply the nonlinear layer to only a part of the state, usually a single element, whereas external rounds (also called full rounds) process all elements in the same way.
More precisely, \PoseidonTwoPi processes an internal state $x = (x_0,\cdots,x_{t-1}) \in \mathbb{F}_p^t$ as follows:
\begin{align*}
 \mathsf{Poseidon2}^{\pi}(x) = \mathcal{E}_{R_F-1} \circ \cdots \circ \mathcal{E}_{R_F/2} \circ \mathcal{I}_{R_P-1} \circ \cdots \circ \mathcal{I}_{0}\circ \mathcal{E}_{R_F/2-1} \circ \cdots \circ \mathcal{E}_{0}(M_{\mathcal{E}} \cdot x) \,
\end{align*}
where $\mathcal{E}$ and $\mathcal{I}$ refer to external and internal round functions iterated for $R_F$ and $R_P$ rounds, respectively.
Note that a linear layer is applied before running the first external round, which differs from the original \PoseidonPi design.
The external/full round function is defined by:
\begin{align*}
 \mathcal{E}(x) = M_{\mathcal{E}} \cdot \Big(\big(x_0+c_0^{(i)}\big)^d,\cdots,\big(x_{t-1}+c_{t-1}^{(i)}\big)^d\Big) \,
\end{align*}
where $d \geq 3$ is the smallest integer such that gcd$(d,p-1) = 1$, $M_{\mathcal{E}}$ is a $t \times t$ maximum distance separable (MDS) matrix and $c_j^{(i)}$ is the $j$-th round constant for the $i$-th external round.
The internal/partial round function is defined by:
\begin{align*}
 \mathcal{I}(x) = M_{\mathcal{I}} \cdot \Big(\big(x_0+\hat{c}_0^{(i)}\big)^d,x_1,\cdots,x_{t-1}\Big) \,
\end{align*}
where $d \geq 3$ as before, $M_{\mathcal{I}}$ is a $t \times t$ MDS matrix and $\hat{c}_0^{(i)}$ is the round constant for the $i$-th internal round.


\paragraph{Efficient instantiations for hash-based signatures over MPC.}
Since \PoseidonTwo is a generic construction, all instantiations do not provide the same level of MPC-friendliness.
Because all operations except exponentiations can be computed locally in an MPC setting, one should aim to minimize the $d$ parameter as it would reduce the number of multiplications.
Because the number of exponentiations is directly determined by the $t$ and $R = R_F + R_P$ parameters, it is natural to seek to minimize their values.
However, at the hash function level, selecting the optimal parameters depends on the size of the input to be processed.
For large inputs that require a sponge mode as the underlying construction, having a large rate would allow to absorb more data per permutation, and eventually leading to fewer calls and fewer exponentiations in the end.
In the case of hash-based signatures, most hash calls process small inputs to compute either hash chains from secret keys or nodes in the Merkle tree, with the exception of leafs which are obtained by hashing multiple public keys.
This is why the generalized \XMSS scheme from~\cite{cryptoeprint:2025/055} instantiates \PoseidonTwo with the compression mode for chain and tree hashing, whereas it uses the sponge mode for leaf hashing.
For hash-based signatures over MPC however, one can disregard the specific case of leaf and tree hashing: since all inputs are public it is possible to recombine the shared values together to run the calculations in a non-distributed manner.
Therefore, the rest of this document focuses on hash chains using \PoseidonTwo with the compression mode and $t = 16$ over a 31-bit prime field for efficient SNARK-based aggregation, as instantiated in~\cite{cryptoeprint:2025/055}.
We considered three different 31-bit prime fields which allow for efficient arithmetic, namely Mersenne31, KoalaBear and BabyBear.
To compare the MPC-friendliness of the corresponding \PoseidonTwo instantiations, we do not consider the number of multiplications per se but the multiplicative depth instead.
Indeed, since parallel (\textit{i.e.}, independent)  multiplications can be processed within the same communication round, minimizing the multiplicative depth is key when latency is a concern, as in distributed systems.
As reported in Table~\ref{tab:poseidon2_inst31}, even though the KoalaBear prime field leads to the instantiation with the highest number of rounds, it is nevertheless the most MPC-friendly\footnote{KoalaBear and BabyBear primes also show advantages over Mersenne31 when it comes to SNARKS thanks to their two-adic multiplactive subgroups for Cooley-Tukey NTTs.} thanks to its minimal $d$ value, both in terms of multiplicative depth and precomputed data.
Therefore, throughout the rest of this document, we focus on instantiations over the KoalaBear prime field for optimal online performance.

\renewcommand\arraystretch{1.25}
\begin{table}[t]
	\centering
  	\resizebox{\textwidth}{!}{
	\begin{tabular}{cccccccc}
		\toprule
    		\textbf{Prime} & \multicolumn{4}{c}{\textbf{Parameters}} & \textbf{Sbox impl.} & \multicolumn{2}{c}{\textbf{MPC metrics}}\\
    		 & {$t$} & {$d$} & {$R_F$} & {$R_P$} & & depth & triples \\
    		\midrule
    		Mersenne31 ($2^{31}-1$) & 16 & 5 & 8 & 14 & $(x^2)^2 \cdot x$  & 66 & 426 \\
    		BabyBear  ($2^{31}-2^{27} + 1$) & 16 & 7 & 8 & 13 & $(x^2)^3 \cdot x$  & 63 & 564 \\
    		KoalaBear ($2^{31}-2^{24}+1$) & 16 & 3 & 8 & 20 & $x^3$ & 28 & 296 \\
    		\bottomrule
	\end{tabular}
	}
	\caption{\PoseidonTwoPi parameters for 31-bit prime fields. The reported results assume that cube evaluations have a multiplicative depth of 1 using the technique from~\cite{10.1145/2976749.2978332} which requires 2 precomputed triples (1 Beaver + 1 cube).\label{tab:poseidon2_inst31}}
\end{table}

\subsection{MPC protocols}
MPC protocols greatly differ based on their security, adversarial and network assumptions. This section aims at identifying the relevant properties to target in the case of distributed validators.
\paragraph{Adversarial structure.}
Let $n$ denote the number of participating parties and let $t$ denote a bound on the number of parties that may be corrupted.
MPC protocols are usually designed to provide security in either the dishonest (\textit{i.e.}, $t < n$), honest (\textit{i.e.}, $t < n/2$) or two-thirds honest (\textit{i.e.}, $t < n/3$) majority settings.
While dishonest-majority protocols offer the strongest security guarantees, they come at a significant cost.
Restricting the model to an honest majority, however, enables substantial performance improvements.
Thus, it is crucial to determine whether a dishonest-majority setting is truly necessary to achieve optimal efficiency.
Regarding distributed validators, it is important to note that they rely not only on a threshold signature scheme (TSS) but also on a consensus algorithm\footnote{\url{https://github.com/ethereum/distributed-validator-specs}}, which ensures that all parties agree on the same message to be  signed.
It is well known that, in an asynchronous network, Byzantine fault tolerance (BFT) can only be achieved if $t < n/3$~\cite{10.1145/322186.322188}.
Therefore, considering a two-thirds honest majority for the MPC computation of the signature aligns well with BFT assumptions.
%However, if crash fault tolerance (CFT) is sufficient instead of BFT, the fault tolerance threshold may increase and considering an honest majority would allow the system to tolerate more faulty parties.
 
\paragraph{Adversarial behaviour.}
In the MPC litterature, corrupted parties are either considered malicious/active if they behave arbitrarily (\textit{i.e.}, they can deviate from the protocol) or semi-honest/passive if they follow the protocol but combine their respective information to learn more than they should be allowed to.
To ensure general compatibility with BFT consensus protocols, we aim for malicious security.
Still, protocols in the semi-honest model can be of interest for efficiency if crash fault tolerance (CFT) suffices.

\paragraph{Security assumption.}
The security of an MPC protocol can rely on different assumptions. It can be \textit{computational} secure, meaning that it depends on the hardness of specific mathematical problems (\textit{e.g.}, factoring large numbers), or it can be \textit{information-theoretic} secure, meaning that it achieves security based on principles of information theory without relying on computational hardness.
Since our end goal is to compute a post-quantum signature scheme in an MPC fashion, we aim for information-theoretic secure MPC protocols as they are not threatened by quantum computers.
%While the information-theoretic setting provides the highest security, it is not always possible to achieve and requires an honest majority.