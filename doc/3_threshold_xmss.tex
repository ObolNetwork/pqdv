\section{Towards threshold XMSS}
The downside of building a threshold hash-based signature by leveraging a SNARK system as mentioned above is that the aggregation of threshold signatures would not be straightforward (since threshold signatures are proofs instead of raw hash-based signatures).
In the case of the Beacon chain where threshold signatures occur \textit{before} aggregation duties, it is imperative for distributed validator middlewares to output signatures that can be aggregated according to the protocol.
Therefore, this section focuses on constructions which lead to threshold Winternitz signatures that are indistinguishable from non-threshold ones, as in \BLS.
\subsection{Distributed hash-based signatures with Boolean shares}
Distributed variants of hash-based signatures, including \XMSS, have been explored by Kelsey, Lang and Lucks in~\cite{cryptoeprint:2022/241} where they introduce $n$-of-$n$ and $k$-of-$n$ threshold signature schemes which rely on Boolean shares.
%\paragraph{$n$-of-$n$ setting.}
For the $n$-of-$n$ setting, a trusted dealer starts from an existing Merkle tree and splits each \WOTS secret key $\mathsf{sk_i}$ by generating $n$ random values $\mathsf{r}^0_i,\cdots,\mathsf{r}^{n-1}_i$ to compute $\mathsf{r}^h_i = \mathsf{r}^0_i \oplus \mathsf{r}^1_i \oplus \cdots \oplus \mathsf{r}^{n-1}_i \oplus \mathsf{sk}_i$.
This introduces an additional party called the \textit{helper} whose role is to store and provide the relevant helper shares whenever required.
That way, to produce a \WOTS using for $\mathsf{sk}_i$, each party can sign independently using its Boolean key share assuming the aggregator has access to the helper share $r^h_i$.
Note that it has to be done for each component of the secret key: assuming a \WOTS scheme to sign $n=vw$-bit messages, it means that each \WOTS key requires $v2^w-1$ helper shares.
Furthermore, the trusted dealer also needs to provide the helper with  shares for each Merkle paths, leading to high memory requirement for the helper overall.
%\mto generate random values $n$ random values and to exclusive-OR them
To minimize memory usage for the parties, the key shares are actually generated pseudorandomly using a \PRF as detailed in Algorithm~\ref{alg:bool_split_n}.
To turn their $n$-of-$n$ scheme into a $k$-of-$n$ threshold scheme, they propose to instantiate a Merkle tree that contains keys for all possible $\binom{n}{k}$ quorums.
Beyond complexity, this increases the height of the Merkle tree and hence the signature size as well as the memory requirements for the helper.
Overall, this approach comes with several limitations from a DVT perspective.
First, it is incompatible with distributed key generation (DKG) algorithms since a trusted dealer is required to split the key into multiple shares.
While supporting DKG is not a necessary prerequisite for distributed validators, it is a valuable feature as it ensures that the private key is never known in its entirety by any single party.
Second, and more importantly, the helper role contradicts with the nature of DVT by introducing a single point of failure which affects decentralization.
Therefore, we investigate alternative solutions that could overcome these weaknesses.


%\begin{myalgorithm}{Split a Merkle tree of WOTS keys into distributed key shares for $n$-of-$n$ signatures, according to~\cite{cryptoeprint:2022/241}.}

\begin{algorithm}[htbp]
    \caption{Split a Merkle tree of \WOTS keys into distributed key shares for $n$-of-$n$ signatures, according to~\cite{cryptoeprint:2022/241}.}
    \begin{algostyle} % Apply styling box inside
Input parameters:
\vspace{-.75em}
\begin{itemize}
\itemsep-.5em
\item Merkle tree built out of a $n$-bit hash function $H$ and $2^h$ \WOTS secret keys $\mathsf{sk}_0,\cdots,\mathsf{sk}_{2^h-1}$ to sign $n=vw$-bit messages (\textit{i.e.}, $\mathsf{sk}_i = (\mathsf{sk}_{i,0}, \cdots, \mathsf{sk}_{i,v-1})$).
\item A pseudorandom function $\mathsf{PRF}_K(x,l)$ parametrized by a $k$-bit key $K$ which takes as input a seed $x$ along with the output bit length $l$.
\item A set of distributed parties $\mathcal{P}$.
\end{itemize}

Output parameters:
\vspace{-.75em}
\begin{itemize}
\itemsep-.5em
\item Secret keys $\mathsf{key}_p$ for each party $p \in \mathcal{P}$.
\item Helper shares $\mathsf{sk}^h_{i,j}$ and $\mathsf{path}^h_{i}$ for $i \in \{0,\cdots,2^h-1\}$ and $j \in \{0, \cdots, v-1\}$.
\end{itemize}

\tcp{picks secrets at random for each party}
\ForEach(){$p \in \mathcal{P}$} {
	$\mathsf{key}_p \xleftarrow[]{\$} \{0,1\}^k $ 
}\vspace{1em}

%\For(\tcp*[f]{for each leaf}){$j=0$ \KwTo $2^h$} {
\tcp{builds Merkle path helper shares}
\For(){$i=0$ \KwTo $2^h-1$} {
	$\mathsf{path}^h_i \leftarrow \mathsf{path}_i$\\
	\ForEach(){$p \in \mathcal{P}$} {
		$\mathsf{path}^h_i \leftarrow \mathsf{path}^h_i \oplus \mathsf{PRF}_{\mathsf{key}_p}\big((\mathsf{domain_{path}}, i), nh\big)$
	}
}\vspace{1em}
	
\tcp{builds WOTS key helper shares}
\For(\tcp*[f]{for each WOTS secret key}){$i=0$ \KwTo $2^h-1$} {
	\For(\tcp*[f]{for each key component}){$j=0$ \KwTo $v-1$} {
		\For(\tcp*[f]{for each w-bit chunk}){$c=0$ \KwTo $2^w-1$} {
		$\mathsf{sk}^h_{i,j}[c] \leftarrow H^c(\mathsf{sk}_{i,j})$ \\
		\ForEach(){$p \in \mathcal{P}$} {
			$\mathsf{sk}^h_{i,j}[c] \leftarrow \mathsf{sk}^h_{i,j}[c] \oplus \mathsf{PRF}_{\mathsf{key}_p}\big((\mathsf{domain_{key}}, i, j, c), n\big)$
			}
		}
	}

}
\label{alg:bool_split_n}
\end{algostyle}
\end{algorithm}

\subsection{Leveraging secret sharing}
Another approach is to leverage a secret sharing scheme to split \WOTS secret keys into shares distributed among participants.
However, it requires to jointly compute all hash function calls in a multi-party computation (MPC) setting.
%However, the difficulty arises when Merkle trees come into play: computing Merkle paths from secret key shares requires to process all hash function calls in a multi-party computation (MPC) fashion.
This can be very challenging in practice, especially in low latency scenarios such as performing validator duties on Ethereum, as the time to produce a threshold signature may largely exceed the requirements (see \textit{e.g.} the work from Cozzo and Smart~\cite{sharing_luov19} which estimates around 85 minutes to compute a threshold \textsf{SPHINCS+} signature with \textsf{SHA-3} as the underlying hash function).
A possible workaround could be to store all secret keys calculated during key generation, so that it is possible to sign messages efficiently without any online MPC calculation.
However, as highlighted in Table~\ref{tab:poseidon2_inst31}, this could lead to unrealistic memory usage as it requires to precompute secret keys for every possible chunk value, and this for all tree leafs.
Since tree nodes are not considered secret material thanks to the preimage resistance of the underlying hash function, a more pragmatic approach would be to only store the plain (\textit{i.e.}, non-shared) leafs value so that signers will be able to calculate Merkle paths in a non-distributed manner when generating signatures.
Nevertheless, in the case of DKGs, this still requires computing all hash function calls over MPC at key generation time.
A comprehensive performance analysis of MPC hash functions is necessary to evaluate the timing constraints of DKG setups and to identify practical time-memory tradeoffs for signature generation.


%Assuming a Merkle tree instantiated with a 256-bit hash function and $2^{18}$ \WOTS keys with a parameter $w = 4$, that would lead to a memory requirement of $\approx 4.5$ gigabytes


%Assuming a Merkle tree instantiated with a 256-bit hash function and $2^{18}$ \WOTS keys with a parameter $w = 4$, that would lead to a memory requirement of $\approx 512$ MiB.
%Note that this estimate does not take into account the storage of Merkle paths.

%Again, to avoid joint hash calculations on-the-fly, all Merkle tree leafs could be precomputed during key generation.
%Because they are not considered secret material thanks to the preimage resistance of the underlying hash function, the parties can recombine their shares to get the plain (\textit{i.e.}, non-shared) leafs values so that they will be able to calculate Merkle paths in a non-distributed manner when generating signatures.


%Still assuming a 256-bit hash function, this would further increase the memory requirements by $8$ MiB, leading to $520$ MiB overall.

%Note that this would require to be the done for all Merkle paths as well.
%Assuming a Merkle tree instantiated with a 256-bit hash function and $2^{18}$ \WOTS keys with a parameter $w = 4$, that would lead to a memory requirement of $\approx 4.5$ gigabytes.
%While this might be acceptable, it remains unclear what would be the (supposedly colossal) timing requirements for the key generation.
%To answer this question, we hereafter review the performance of MPC-friendly hash functions to assess what time-memory tradeoffs would be of practical interest.
%More generally, a comprehensive performance analysis of MPC hash functions is crucial to assess what time-memory tradeoffs would be of practical interest.



%Bear in mind that avoiding any MPC hash invocation implies that \WOTS secret keys cannot be generated pseudo-randomly from a secret seed shared amongst participants, further increasing memory consumption.
 
%Still, even if all parties store their own \WOTS secret key shares and corresponding Merkle paths, they cannot sign non-interactively.
%Indeed, \XMSS not being a deterministic signature scheme, the signers have to agree on a random number to be included in the randomized message digest calculation.


%\begin{algorithm}[htbp]
%    \caption{Threshold \XMSS key generation by a trusted dealer.}
%    \begin{algostyle} % Apply styling box inside
%Input parameters:
%\vspace{-.75em}
%\begin{itemize}
%\itemsep-.5em
%\item Merkle tree built out of a $n$-bit hash function $H$ and $2^h$ \WOTS secret keys $\mathsf{sk}_0,\cdots,\mathsf{sk}_{2^h-1}$ to sign $n=vw$-bit messages (\textit{i.e.}, $\mathsf{sk}_i = (\mathsf{sk}_{i,0}, \cdots, \mathsf{sk}_{i,v-1})$).
%\item An algorithm \textsf{Split} which 
%\item 
%\item A set of distributed parties $\mathcal{P}$.
%\end{itemize}
%
%Output parameters:
%\vspace{-.75em}
%\begin{itemize}
%\itemsep-.5em
%\item Secret keys $\mathsf{key}_p$ for each party $p \in \mathcal{P}$.
%\item Helper shares $\mathsf{sk}^h_{i,j}$ and $\mathsf{path}^h_{i}$ for $i \in \{0,\cdots,2^h-1\}$ and $j \in \{0, \cdots, v-1\}$.
%\end{itemize}
%
%\tcp{builds WOTS keys}
%\For(\tcp*[f]{for each WOTS secret key}){$i=0$ \KwTo $2^h-1$} {
%	\For(\tcp*[f]{for each key component}){$j=0$ \KwTo $v-1$} {
%		$\mathsf{sk}_{i,j} \xleftarrow[]{\$} \{0,1\}^n $  \\
%		$\mathsf{pk}_{i,j} \leftarrow H^{2^w-1}(\mathsf{sk}_{i,j})$  \\
%		\ForEach(){$p \in \mathcal{P}$} {
%			$\mathsf{sk}^h_{i,j}[c] \leftarrow \mathsf{sk}^h_{i,j}[c] \oplus \mathsf{PRF}_{\mathsf{key}_p}\big((\mathsf{domain_{key}}, i, j, c), n\big)$
%			}
%	}
%
%}
%\label{alg:bool_split_n}
%\end{algostyle}
%\end{algorithm}